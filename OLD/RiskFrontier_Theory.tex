% RiskFrontier_Theory.tex
\documentclass[11pt,a4paper]{article}

% --------- Pacotes básicos ----------
\usepackage[brazil]{babel}
\usepackage[utf8]{inputenc}
\usepackage[T1]{fontenc}
\usepackage{lmodern}
\usepackage{microtype}
\usepackage{amsmath, amssymb, amsfonts, bm, mathtools}
\usepackage{graphicx}
\usepackage{booktabs}
\usepackage{xcolor}
\usepackage{geometry}
\usepackage{hyperref}
\usepackage{fancyhdr}
\usepackage{enumitem}

\geometry{margin=2.6cm}

% Permite que \includegraphics tente png/jpg/pdf nessa ordem
\DeclareGraphicsExtensions{.png,.jpg,.jpeg,.pdf}

% --------- Cores institucionais ----------
\definecolor{safraBlue}{HTML}{002855}
\definecolor{safraGold}{HTML}{C6A664}

% --------- Links discretos ----------
\hypersetup{
  colorlinks=true,
  linkcolor=safraBlue,
  urlcolor=safraBlue,
  citecolor=safraBlue
}

% --------- Macro do logo com fallback (PNG > JPG > nada) ----------
% --------- Macro do logo com fallback (PDF > PNG > JPG) ----------
\newcommand{\BankLogo}{%
  \IfFileExists{logoEnf.pdf}{\includegraphics[height=1.2cm]{logoEnf.pdf}}{%
    \IfFileExists{logoEnf_fixed.png}{\includegraphics[height=1.2cm]{logoEnf_fixed.png}}{%
      \IfFileExists{logoEnf.png}{\includegraphics[height=1.2cm]{logoEnf.png}}{%
        \IfFileExists{logoEnf.jpg}{\includegraphics[height=1.2cm]{logoEnf.jpg}}{}%
      }%
    }%
  }%
}

% --------- Cabeçalho com logo ----------
\pagestyle{fancy}
\fancyhf{}
\lhead{\BankLogo}
\rhead{\textcolor{safraBlue}{\textbf{Risk \& Return Analyzer}}}
\cfoot{\thepage}

% --------- Título ----------
\title{\textbf{Credit Portfolio Risk \& Return Analyzer}\\[2mm]
\large \textcolor{safraBlue}{Financial Engineering for Bankers}}
\author{\textbf{Walter C Neto}}
\date{\today}

% --------- Comandos úteis ----------
\newcommand{\PD}{\mathrm{PD}}
\newcommand{\LGD}{\mathrm{LGD}}
\newcommand{\EAD}{\mathrm{EAD}}
\newcommand{\EL}{\mathrm{EL}}
\newcommand{\UL}{\mathrm{UL}}
\newcommand{\Kcap}{\mathrm{K}}
\newcommand{\Rho}{\rho}
\newcommand{\VaR}{\mathrm{VaR}}
\newcommand{\ES}{\mathrm{ES}}
\newcommand{\bp}{\mathrm{bp}}
\newcommand{\Phiinv}{\Phi^{-1}}

\begin{document}
\maketitle
\vspace{-1.0em}
\hrule
\vspace{1.2em}

\section{Visão geral}
Este documento descreve a fundamentação teórica e as fórmulas utilizadas no aplicativo \emph{Credit Portfolio Risk \& Return Analyzer}. O objetivo é analisar uma carteira de crédito sob a ótica de risco e retorno, estimando:
\begin{itemize}[leftmargin=1.2em]
  \item o \textbf{required spread} (cobertura de \(\EL\), custo de capital via \(\Kcap\), funding e opex),
  \item o \textbf{mispricing} percentual por exposição,
  \item a \textbf{risk contribution} em pontos--base,
  \item e os efeitos de \textbf{rebalanceamentos} (vender piores / aumentar melhores).
\end{itemize}

As entradas principais por exposição são: \(\EAD\), \(\PD\) (\%), \(\LGD\) (\%), spread observado em \(\bp\), opcionalmente prazo (\emph{maturity}) em anos e correlação \(\Rho\).

\section{Normalização dos dados}
O aplicativo normaliza nomes de colunas (sinônimos) e converte números no formato local (ex.: ``10,5\%'' \(\rightarrow\) \(0{,}105\)). As grandezas percentuais são transformadas para fração:
\begin{equation}
\PD_f = \frac{\PD}{100}, \qquad \LGD_f = \frac{\LGD}{100}.
\end{equation}

\section{Modelo ASRF (Vasicek de um fator)}
O cálculo de capital segue o paradigma \textbf{ASRF} (\emph{Asymptotic Single Risk Factor}). Para cada exposição \(i\), o valor latente de ativo \(A_i\) é:
\begin{equation}
A_i = \sqrt{\Rho}\,Y + \sqrt{1-\Rho}\,\varepsilon_i,
\end{equation}
onde \(Y \sim \mathcal{N}(0,1)\) é o fator sistêmico e \(\varepsilon_i \sim \mathcal{N}(0,1)\) é o idiossincrático. O default ocorre quando \(A_i \leq \Phi^{-1}(\PD_f)\), com \(\Phi\) a CDF normal padrão.

\subsection{Probabilidade condicional de default}
Condicionado em \(Y=y\), a probabilidade de default é:
\begin{equation}
\PD_{\text{cond}}(y) \;=\; \Phi\!\left(\frac{\Phi^{-1}(\PD_f) - \sqrt{\Rho}\,y}{\sqrt{1-\Rho}}\right).
\end{equation}

\subsection{Capital regulatório em nível de confiança}
Para um nível de confiança \(\alpha\) (tipicamente \(99{,}9\%\)), toma-se \(y_\alpha = \Phi^{-1}(1-\alpha)\) (quantil adverso do fator). Assim, a perda condicional é aproximada por:
\begin{equation}
\mathrm{Loss}_\alpha \;\approx\; \LGD_f \cdot \PD_{\text{cond}}(y_\alpha).
\end{equation}
O \textbf{capital condicional} (por unidade de exposição) é então:
\begin{equation}
\Kcap \;=\; \max\!\Big(\LGD_f \cdot \Phi\!\Big(\frac{\Phi^{-1}(\PD_f) + \sqrt{\Rho}\,\Phi^{-1}(\alpha)}{\sqrt{1-\Rho}}\Big) - \PD_f \cdot \LGD_f,\ 0\Big).
\label{eq:K}
\end{equation}
No app adota-se \(\alpha = 0{,}999\), logo \(\Phi^{-1}(\alpha)\approx 3{,}090\).

\subsection{Ajuste de maturidade (exposições corporativas)}
Quando há \emph{maturity} \(M\) (anos), aplica-se o multiplicador:
\begin{align}
b(\PD_f) &= \big(0{,}11852 - 0{,}05478\cdot \ln(\PD_f)\big)^2,\\
\mathrm{MA}(M,\PD_f) &= \frac{1 + (M-2{,}5)\,b(\PD_f)}{1 - 1{,}5\, b(\PD_f)}.
\end{align}
O capital ajustado torna-se \(\Kcap^\star = \Kcap \cdot \mathrm{MA}\).

\paragraph{Observação sobre \(\Rho\).} No app, \(\Rho\) pode vir do arquivo de entrada (coluna \texttt{rho}) ou ser fixado por parâmetro (\emph{slider}). Diferentemente de certas fórmulas regulatórias onde \(\Rho\) é função de \(\PD\), aqui usa-se o valor informado/selecionado.

\section{Perda Esperada, Capital e \emph{Required Spread}}
A \textbf{perda esperada} por unidade de exposição é:
\begin{equation}
\EL = \PD_f \cdot \LGD_f.
\end{equation}
O app calcula o \textbf{required spread} (em \(\bp\)) como:
\begin{equation}
\mathrm{RequiredBP} \;=\; 10{,}000 \cdot \Big(\EL + \text{Hurdle}\cdot \Kcap^\star\Big) + \mathrm{FundingBP} + \mathrm{OpexBP}.
\label{eq:reqbp}
\end{equation}
Isto cobre a perda esperada, um prêmio pelo custo de capital (via \(\Kcap^\star\)), e \emph{overheads} de funding/opex, todos convertidos para pontos--base.

\section{Mispricing, Risk Contribution e Métricas de Portfólio}
\subsection{Mispricing (\%)}
Comparando o spread observado \(\mathrm{SpreadBP}\) com o requerido:
\begin{equation}
\mathrm{Mispricing\%} \;=\; \frac{\mathrm{SpreadBP} - \mathrm{RequiredBP}}{\max(\mathrm{RequiredBP},\,10^{-9})}.
\end{equation}
Valores positivos sugerem sobrepreço (\emph{over-earning}); negativos, subpreço (\emph{under-earning}).

\subsection{Risk contribution em bp}
Para análise bidimensional, considera-se a \textbf{risk contribution} em bp por exposição como:
\begin{equation}
\mathrm{RC}_{\bp}^{(i)} = 10{,}000 \cdot \Kcap_i^\star.
\end{equation}
No agregado, a métrica síntese usada no app é a média ponderada por \(\EAD\):
\begin{equation}
\overline{\mathrm{RC}}_{\bp} \;=\; \sum_i w_i \cdot (10{,}000\,\Kcap_i^\star), \qquad
w_i = \frac{\EAD_i}{\sum_j \EAD_j}.
\end{equation}

\subsection{Tabela de desempenho do portfólio}
Para o portfólio (original, após vender piores \(N\), e após aumentar melhores \(N\)):
\begin{align}
\textbf{Exposure Amount (MM)} &: \ \frac{\sum_i \EAD_i}{10^6},\\
\textbf{Total Spread (bp)} &: \ \sum_i w_i \cdot \mathrm{SpreadBP}_i,\\
\textbf{Expected Spread (bp)} &: \ \sum_i w_i \cdot \mathrm{RequiredBP}_i,\\
\textbf{Unexpected Loss (MM)} &: \ \frac{\sum_i \EAD_i \cdot \Kcap_i^\star}{10^6}.
\end{align}
Define-se ainda um índice tipo \emph{Sharpe}:
\begin{equation}
\mathrm{SharpeLike} \;=\; \frac{\sum_i w_i \cdot \mathrm{RequiredBP}_i}{\sum_i w_i \cdot (10{,}000\,\Kcap_i^\star)}.
\end{equation}
Ele compara ``retorno requerido'' por unidade de risco (capital) em bp. No app, é apresentado em \%.

\section{Rebalanceamentos}
\subsection{Vender piores \(N\)}
Ordena-se por \(\mathrm{Mispricing\%}\) ascendente e remove-se as \(N\) piores exposições (subprecificadas). Recalcula-se a tabela de portfólio.

\subsection{Aumentar melhores \(N\)}
Ordena-se por \(\mathrm{Mispricing\%}\) descendente e multiplica-se a \(\EAD\) das \(N\) melhores por um fator \(F>1\). Recomputam-se as métricas.%
\footnote{No app, o fator \(F\) é controlado por \emph{slider}.}

\section{Modo rápido (heurístico)}
Como aproximação expedita, o app oferece um modo \emph{heurístico} para o capital:
\begin{equation}
\Kcap_{\text{heur}} \;\approx\; \LGD_f \cdot \sqrt{\PD_f (1-\PD_f)} \cdot \sqrt{1+\Rho}.
\end{equation}
Essa forma não é regulatória, mas serve como \emph{proxy} de risco para análise exploratória quando não se deseja o cálculo ASRF completo.

\section{Unidades e cuidados práticos}
\begin{itemize}[leftmargin=1.2em]
  \item \(\PD\) e \(\LGD\) são fornecidas em \%, mas o cálculo utiliza as frações \(\PD_f\) e \(\LGD_f\).
  \item \(\mathrm{SpreadBP},\ \mathrm{FundingBP},\ \mathrm{OpexBP}\) estão em \(\bp\) (pontos--base).
  \item \(\Kcap^\star\) é adimensional (perda relativa por unidade de exposição).
  \item Em \eqref{eq:reqbp}, o fator \(10{,}000\) converte uma taxa em \% para \(\bp\).
  \item Para estabilidade numérica, o app satura \(\PD_f,\LGD_f,\Rho\) dentro de intervalos \([10^{-6},0.999999]\).
\end{itemize}

\section{Glossário de símbolos}
\begin{center}
\begin{tabular}{@{}ll@{}}
\toprule
Símbolo & Descrição \\
\midrule
\(\EAD\) & \emph{Exposure at Default} (exposição) \\
\(\PD\) (\%) & Probabilidade de default (entrada em \%) \\
\(\LGD\) (\%) & Perda dado default (entrada em \%) \\
\(\PD_f,\LGD_f\) & Frações (0--1) usadas no cálculo \\
\(\Rho\) & Correlação ativo--fator sistêmico \\
\(\Kcap\) & Capital por unidade de exposição (ASRF) \\
\(\EL\) & Perda esperada \\
\(\mathrm{RequiredBP}\) & Spread requerido em pontos--base \\
\(\mathrm{SpreadBP}\) & Spread observado em pontos--base \\
\(\mathrm{Mispricing\%}\) & Sinaliza sobre/subprecificação \\
\(\mathrm{RC}_{\bp}\) & Risk contribution em bp \\
\bottomrule
\end{tabular}
\end{center}

\section{Limitações e extensões}
\begin{itemize}[leftmargin=1.2em]
  \item ASRF assume portfólio infinitamente granular e um único fator sistêmico.
  \item A escolha de \(\Rho\) afeta fortemente \(\Kcap\); no app, \(\Rho\) pode ser inserida ou ajustada via parâmetro.
  \item \emph{Funding} e \emph{Opex} são insumos de negócio, não estimados pelo modelo.
  \item Extensões naturais: \(\VaR\)/\(\ES\) por simulação, \(\Rho\) setorial, correções de concentração, dependência de maturidade e garantias.
\end{itemize}

\vspace{1em}
\noindent\textbf{Observação prática (app).} O gráfico principal utiliza no eixo \(x\) a \(\mathrm{RC}_{\bp}\) por exposição e no eixo \(y\) o spread (observado ou requerido). Uma reta de inclinação mediana (\emph{Sharpe Line}) ajuda a comparar razão retorno/risco relativa.

\vfill
\begin{center}
\textit{Este documento acompanha o aplicativo para fins de entendimento e validação das fórmulas e hipóteses.}
\end{center}

\end{document}